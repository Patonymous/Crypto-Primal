\documentclass[11pt, a4paper]{article}

% Pakiety podstawowe i językowe
\usepackage[utf8]{inputenc}
\usepackage[T1]{fontenc}
\usepackage[polish]{babel}
\usepackage{geometry}
\geometry{margin=2.5cm}

% Pakiety matematyczne
\usepackage{amsmath}
\usepackage{amsfonts}
\usepackage{amssymb}
\usepackage{amsthm}

% Pakiety do formatowania kodu i linków
\usepackage{listings}
\usepackage{xcolor}
\usepackage{hyperref}
\usepackage{graphicx}
\usepackage{float}

% Konfiguracja wyświetlania kodu
\definecolor{codegreen}{rgb}{0,0.6,0}
\definecolor{codegray}{rgb}{0.5,0.5,0.5}
\definecolor{codepurple}{rgb}{0.58,0,0.82}
\definecolor{backcolour}{rgb}{0.95,0.95,0.92}

\lstdefinestyle{mystyle}{
    backgroundcolor=\color{backcolour},   
    commentstyle=\color{codegreen},
    keywordstyle=\color{magenta},
    numberstyle=\tiny\color{codegray},
    stringstyle=\color{codepurple},
    basicstyle=\ttfamily\footnotesize,
    breakatwhitespace=false,         
    breaklines=true,                 
    captionpos=b,                    
    keepspaces=true,                 
    numbers=left,                    
    numbersep=5pt,                  
    showspaces=false,                
    showstringspaces=false,
    showtabs=false,                  
    tabsize=2,
    inputencoding=utf8,
    extendedchars=true,
    literate={ą}{{\k{a}}}1 {ć}{{\'c}}1 {ę}{{\k{e}}}1 {ł}{{\l{}}}1 {ń}{{\'n}}1 {ó}{{\'o}}1 {ś}{{\'s}}1 {ź}{{\'z}}1 {ż}{{\.z}}1
}

\lstset{style=mystyle}

% Tytuł
\title{\textbf{Dokumentacja Projektu: Atak Primal na LWE}\\ Implementacja i analiza kryptosystemu Regev}
\author{}
\date{\today}

\begin{document}

\maketitle
\tableofcontents
\newpage

\section{Wstęp}
Niniejszy dokument opisuje implementację oraz analizę bezpieczeństwa kryptosystemu opartego na problemie \textit{Learning With Errors} (LWE). Projekt zawiera implementację schematu szyfrowania Regeva oraz realizację ataku \textit{Primal Attack} wykorzystującego redukcję krat (LLL i BKZ).

Celem projektu jest demonstracja podatności instancji LWE na ataki kratowe przy nieodpowiednim doborze parametrów oraz analiza wpływu rozmiaru bloku BKZ na skuteczność ataku.

\section{Wymagania i Instalacja}
\label{sec:requirements}

\textbf{Uwaga:} Poprawne uruchomienie programu wymaga specyficznego środowiska ze względu na zależności matematyczne.

\subsection{Wymagania Systemowe}
\begin{itemize}
    \item \textbf{System operacyjny:} Linux (zalecany: Ubuntu/Debian) lub macOS. 
    \textit{Uruchomienie na Windows jest możliwe tylko poprzez WSL (Windows Subsystem for Linux), ponieważ biblioteka \texttt{fpylll} nie posiada oficjalnego wsparcia dla natywnego Windowsa.}
    \item \textbf{Python:} Wersja \textbf{3.10} lub nowsza.
    \textit{Kod źródłowy (plik \texttt{src/printer.py}) wykorzystuje instrukcję \texttt{match ... case}, która została wprowadzona w Pythonie 3.10.}
\end{itemize}

\subsection{Zależności Systemowe (Linux/WSL)}
Przed instalacją bibliotek Pythona należy zainstalować biblioteki systemowe wymagane do obliczeń precyzyjnych i teorii liczb:

\begin{lstlisting}[language=bash, caption=Instalacja bibliotek systemowych (Debian/Ubuntu)]
sudo apt update
sudo apt install build-essential python3-dev libgmp-dev libmpfr-dev
\end{lstlisting}

\subsection{Biblioteki Python}
Projekt wymaga następujących bibliotek zewnętrznych:
\begin{itemize}
    \item \texttt{numpy} -- obliczenia macierzowe.
    \item \texttt{fpylll} -- implementacja algorytmów redukcji krat (LLL, BKZ).
    \item \texttt{matplotlib} -- generowanie wykresów do benchmarków.
\end{itemize}

Instalacja via pip:
\begin{lstlisting}[language=bash, caption=Instalacja zależności Python]
pip install numpy matplotlib fpylll
\end{lstlisting}

\section{Opis Techniczny}

\subsection{Struktura Projektu}
\begin{itemize}
    \item \texttt{main.py} -- Główny punkt wejścia programu (CLI). Obsługuje argumenty i steruje przepływem.
    \item \texttt{src/lwe.py} -- Klasa \texttt{LWE}. Odpowiada za generowanie kluczy ($A, s, e$), obliczanie $b = As + e \pmod q$ oraz zapis/odczyt instancji.
    \item \texttt{src/regev.py} -- Implementacja szyfrowania i deszyfrowania metodą Regeva.
    \item \texttt{src/attack.py} -- Implementacja ataku \textit{Primal}. Konstruuje macierz kraty i uruchamia redukcję.
    \item \texttt{src/interactive.py} -- Tryb edukacyjny pokazujący atak krok po kroku.
    \item \texttt{src/benchmark.py} -- Moduł testowy do analizy wydajności i skuteczności ataku.
\end{itemize}

\subsection{Model Matematyczny i Szyfrowanie}
Instancja LWE zdefiniowana jest przez parametry $(n, m, q, \alpha)$, gdzie:
\begin{itemize}
    \item $n$ -- wymiar sekretu $s \in \mathbb{Z}_q^n$.
    \item $m$ -- liczba próbek (wierszy macierzy $A$).
    \item $q$ -- moduł (liczba pierwsza).
    \item $\alpha$ -- parametr szumu, gdzie $\sigma = \alpha q$.
\end{itemize}

\textbf{Szyfrowanie (Regev):} Wiadomość jest kodowana bit po bicie. Dla bitu $x \in \{0, 1\}$:
$$ c = \left( \sum_{i \in S} b_i \right) + x \cdot \lfloor q/2 \rfloor \pmod q $$
gdzie $S$ jest losowym podzbiorem indeksów wierszy.

\subsection{Atak Primal (Kannan's Embedding)}
Atak polega na sprowadzeniu problemu LWE do problemu najkrótszego wektora (SVP) w kracie. Konstruowana jest macierz bazy $B$ o wymiarach $(m+n+1) \times (m+n+1)$:

$$
B = \begin{pmatrix} 
q I_m & 0 & 0 \\
A^T & I_n & 0 \\
b^T & 0 & 1 
\end{pmatrix}
$$

Poszukiwany jest krótki wektor w kracie, który odpowiada wektorowi błędów i sekretu:
$$ v = (e, s, 1) \quad \text{lub} \quad v \cdot k $$
Po konstrukcji bazy stosowana jest redukcja LLL (Lenstra–Lenstra–Lovász) oraz BKZ (Block Korkine-Zolotarev). Odzyskany wektor jest weryfikowany poprzez sprawdzenie poprawności deszyfrowania.

\section{Instrukcja Użytkownika}

Program uruchamiany jest z linii komend poprzez plik \texttt{main.py}.

\subsection{Składnia}
\begin{lstlisting}[language=bash]
./main.py [akcja] [parametry...]
\end{lstlisting}

\subsection{Dostępne akcje}

\subsubsection{1. Generowanie nowej instancji}
Musi być wykonane jako pierwsze. Tworzy pliki \texttt{lwe.key} (prywatny) i \texttt{lwe.pub} (publiczny).
\begin{lstlisting}[language=bash]
# Składnia: generuj [n] [m] [q] [alpha]
./main.py generuj 10 60 101 0.01
\end{lstlisting}

\subsubsection{2. Szyfrowanie wiadomości}
Szyfruje ciąg znaków używając wygenerowanej instancji. Tworzy plik \texttt{ciphertext.bin}.
\begin{lstlisting}[language=bash]
./main.py szyfruj "TajnyTekst"
\end{lstlisting}

\subsubsection{3. Odszyfrowywanie (Legalne)}
Używa klucza prywatnego do odczytania wiadomości.
\begin{lstlisting}[language=bash]
./main.py odszyfruj
\end{lstlisting}

\subsubsection{4. Atak (Odzyskiwanie sekretu)}
Próbuje odtworzyć sekret $s$ wyłącznie na podstawie klucza publicznego, a następnie odszyfrować wiadomość.
\begin{lstlisting}[language=bash]
# Składnia: atakuj [rozmiar_bloku_BKZ]
./main.py atakuj 25
\end{lstlisting}
\textit{Parametr rozmiar\_bloku\_BKZ (domyślnie 25) wpływa na siłę ataku i czas wykonania.}

\subsubsection{5. Tryb Interaktywny}
Edukacyjny tryb z opisem każdego kroku.
\begin{lstlisting}[language=bash]
./main.py interaktywny
\end{lstlisting}

\subsubsection{6. Benchmark}
Przeprowadza serię testów i generuje wykresy w folderze \texttt{report/img}.
\begin{lstlisting}[language=bash]
# Składnia: benchmark [liczba_prób] [output_dir]
./main.py benchmark 5 report/img
\end{lstlisting}

\section{Rozwiązywanie problemów}
\begin{enumerate}
    \item \textbf{Błąd: `No module named fpylll`} \\
    Program nie może znaleźć biblioteki do krat. Upewnij się, że zainstalowałeś biblioteki systemowe (GMP, MPFR) przed uruchomieniem \texttt{pip install fpylll}. Zobacz sekcję \ref{sec:requirements}.
    
    \item \textbf{Błąd składni w pliku `printer.py`} \\
    Twoja wersja Pythona jest starsza niż 3.10. Zaktualizuj Pythona.
    
    \item \textbf{Atak nie powiódł się} \\
    Dla parametrów domyślnych ($n=10$) atak BKZ-20 lub BKZ-25 powinien być skuteczny. Jeśli atak się nie udaje:
    \begin{itemize}
        \item Zwiększ rozmiar bloku BKZ (np. do 30 lub 35).
        \item Zwiększ liczbę próbek $m$ przy generowaniu instancji (np. $m=70$).
        \item Zmniejsz szum $\alpha$ (np. $0.005$).
    \end{itemize}
\end{enumerate}

\end{document}